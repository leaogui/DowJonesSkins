% Para facilitar a manutenção é sempre melhore criar um arquivo por capitulo, para exemplo isso não é necessário 

%---------------------------------------------------------------------------------------
\chapter{Modelo Teórico e Pressupostos (ou Hipóteses) da Pesquisa}

Em linhas gerais, o objetivo deste projeto é desenvolver uma solução inovadora no mercado 
de venda/troca de skins de CS:GO que implementará um sistema de bolsa de valores e 
Day Trading dessas skins, algo não visto no mercado até então.
Os objetivos principais giram em torno de elaborar e desenvolver um novo sistema 
de venda e troca de skins de jogos eletrônicos, inovando este cenário atual. 
Para isso, será necessário conhecer as soluções existentes no mercado quando falamos em 
aplicações de venda e troca de skins de jogos e entender seu funcionamento como um todo, 
para que identificar os pontos fortes e fracos deste mercado atualmente.
Como objetivos secundários, o projeto visa promover estudo e capacitação sobre tecnologias 
e meios utilizados atualmente para a implementação de aplicações web, além de promover um 
melhor entendimento sobre os ideais de mercado financeiro de ações e o mecanismo de 
Day Trading.


%---------------------------------------------------------------------------------------
\chapter{Métodos de Pesquisa}

 \subsection{Hipóteses}
    No mercado em geral, principalmente no Brasil, não existem sites com a proposta da valorização de skins, muito menos aplicações inspiradas no mundo financeiro de bolsa de valores e Day Trading, então o desenvolvimento do sistema proposto neste documento preencherá esta lacuna fazendo possível um usuário além de fazer as suas trocas, ter uma experiência de bolsa de valores com seus itens do jogo. 
    
    Tendo em vista que grande parte dos jogadores utilizam suas skins também para uso comercial e conseguir dinheiro, foi levantada a hipótese de que o site de valorização tem grande possibilidade de obter popularidade e o agrado do público do jogo envolvido.

\subsection{Metodologia do Projeto}
    Buscando a implementação d aplicação web proposta, este projeto utilizará dos métodos ágeis, principalmente seguindo a figura da metodologia XP. Para auxiliar na execução destes métodos, a ferramenta Pivotal Tracker será utilizada para administrar as histórias de usuários e promover um ambiente para organizá-las e discuti-las.
    
    Como parte dos métodos ágeis, o sistema utilizará duas ferramentas de testes, o Jest e o Cucumber, realizando testes unitários, de integração e testes End-to-End, além do fato de que o Cucumber fará os testes direto nos cenários criados pelas histórias de usuário.
    
    Em questão técnica, a aplicação será feita em cima de um back-end em Node.Js, enquanto o front-end utilizará do HTML e do Handlebars. Para controle de versão, o Git será utilizando, portanto o projeto contém um repositório no GitHub onde cada progresso será salvo, controlando o ciclo de vida do software. Para deploy, o Heroku será utilizado. Com isso, uma aplicação foi criada no Heroku, onde o sistema será posto no ar toda vez que uma mudança for ‘commitada’ no GitHub.
    
    Para estudo de domínio, informações com players ativos de Counter-Strike: Global Offensive serão coletadas, além de informações sobre sites de grande importância no meio de mercado de skins. Como não é uma área documentada, será necessário fazer uma abordagem mais informal e retirar informações direto de clientes e de experiencias próprias dos integrantes deste trabalho.
    
    Pesquisas com alguns jogadores para verificar a viabilidade desta aplicação proposta no projeto serão realizadas.


%\section{Tipo de Pesquisa}
%\lipsum[3-5]

%\section{Plano Amostral (se Pesquisa Quantitativa)}
%\lipsum[3-5]

%\section{Instrumento de Pesquisa e Escalas Utilizadas (Escalas se Pesquisa Quantitativa)}
%\lipsum[3-5]

%\section{Coleta de Dados}
%\lipsum[3-5]

%\section{Análise de Dados}
%\lipsum[3-5]


%---------------------------------------------------------------------------------------
\chapter{Resultados da Pesquisa}

\section{Objetivos Primários}
Os objetivos principais giram em torno de elaborar e desenvolver um novo sistema de venda/troca de skins de jogos eletrônicos, inovando este cenário atual. Para isso, será necessário conhecer as soluções existentes no mercado quando falamos em aplicações de venda/troca de skins de jogos e entender seu funcionamento como um todo, para que identificar os pontos fortes e fracos deste mercado atualmente.

\section{Objetivos Secundários}
Como objetivos secundários, o projeto visa promover estudo e capacitação sobre tecnologias e meios utilizados atualmente para a implementação de aplicações web, além de promover um melhor entendimento sobre os ideais de mercado financeiro de ações e o mecanismo de Day Trading.

\section{Justificativa}
  As principais razões que se levam à execução deste projeto e, por consequência, ao desenvolvimento de aplicações web para troca/venda de skins de jogos eletrônicos são: 

    \begin{itemize}
        \item Inovação do mercado, pois a implementação de um mercado de ações e um Day Trade de skins de CSGO consiste em algo novo para o cenário desenvolvido até então, que permeia sites de mercado comum e aposta;
        \item Exploração de novas tecnologias que farão parte do desenvolvimento do sistema como Handlebars, por exemplo, possibilitando estudo prático para os membros do projeto;
        \item Viabilidade comercial, pois se pode lucrar muito com taxas de trocas, como pode ser vista no artigo de pesquisa da NewZoo sobre o lucro do mercado de jogos em 2018, que passou de 130 bilhões de dólares ao redor do planeta, sendo que o mercado de skins de CSGO movimenta por ano mais de 10 bilhões de dólares, segundo a matéria da ‘The Enemy’.
      \end{itemize}

%---------------------------------------------------------------------------------------
%\chapter{Considerações finais}
%\lipsum[3]

%\section{Resposta à Questão de Pesquisa}
%\lipsum[3-5]

%\section{Objetivos Propostos}
%\lipsum[3-5]

%\section{Contribuições Academicas e Gerenciais}
%\lipsum[3-5]

%\section{Limitações da Pesquisa e Contribuições para Estudo}
%\lipsum[3-5]




