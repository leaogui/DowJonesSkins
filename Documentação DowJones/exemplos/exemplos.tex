\chapter{Exemplos \LaTeX}


% exemplo de como inserir uma referencia adicional no sumario (normalmente não utilizado em um trabalho academico)
\addcontentsline{toc}{chapter}{Exemplos que devem ser lidos :-)}

Esse capítulo tem exemplos de escrita utilizando o \LaTeX  utilizando \abnTeX, é muito simples escrever em \textbf{negrito}, \emph{italico}, ....


Existem diversos tutoriais para uso de \LaTeX, se você está utilizando esse modelo não precisará se preocupar com muitos dos detalhes técnicos do \LaTeX \space e cuidar somente do seu texto.

Escolha seu editor : \url{https://en.wikipedia.org/wiki/Comparison\_of\_TeX\_editors}


\input{exemplos/abnt}

\section{Detalhes textuais}

O documento é dividido em capítulos, e cada capítulo dividido em seções utilizando o \abnTeX \space você pode dividir seus documentos nos níveis a seguir:

\begin{itemize}
\item chapter (1);
\item section (1.1);
\item subsection (1.1.1);
\item subsubsection (1.1.1.1);
\item subsubsubsection (1.1.1.1.1).
\end{itemize}

Tenha em mente que normalmente se utiliza no máximo o nível \emph{subsection}.
Ao definir as divisões do seu trabalho utilizando as diretivas do \LaTeX, elas são automaticamente inseridas no sumário do documento.


\subsection{Caracteres Reservados}



Alguns caracteres são reservados no \LaTeX \space e por isso para utilizar esses caracteres é necessario utilizar uma forma diferenciada de escrita. É possivel utilizar a macro \emph{symbol} com o codigo ascii do caracter desejado.
\begin{itemize}
\item barra invertida : \textbackslash   \symbol{92}    $\backslash$;
\item til  :  \symbol{126} ;
\item cifrão : \$;
\item sublinhado, \emph{underscore}, \emph{underline} : \_;
\item chaves : \} \{.
\end{itemize}

\chapter*[Citações / Referências]{Citações / Referências}

\noindent
    PUC PR. Mercado De Jogos Digitais Cresce No Brasil E No Mundo. G1 Globo, 08/10/2018. Disponível em: \textless https://g1.globo.com/pr/parana/especial-publicitario/puc-pr/profissionais-do-amanha/noticia/2018/10/08/mercado-de-jogos-digitais-cresce-no-brasil-e-no-mundo.ghtml\textgreater. Acesso em: 15 Out. 2019. \\
    
    \noindent
    ITENS ‘COSMÉTICOS’ MOVIMENTAM CULTURA E ECONOMIA DOS JOGOS. E-Arena. Disponível em: \textless https://e-arena.com.br/itens-cosmeticos-movimentam-cultura-e-economia-dos-jogos/\textgreater. Acesso em: 14 Out. 2019. \\
    
    \noindent
    MERCADO DE SKINS DE CS:GO PODE MOVIMENTAR ATÉ 10 BILHÕES DE DÓLARES POR ANO. The Enemy. Disponível em:\textless https://www.theenemy.com.br/esports/csgo-mercado-skins-10-bilhoes-valores-precos\textgreater. Acesso em: 14 Out. 2019. \\
    
    \noindent
    NEWZOO ARTICLES. NewZoo. Disponível em: \textless https://newzoo.com/insights/articles/\textgreater. Acesso em: 15 Out. 2019. \\
    
    \noindent
    SETOR DE GAMES CRESCE ACIMA DA MÉDIA NO PAÍS, MAS É O 13º DO MUNDO. O Tempo. Disponível em: \textless https://www.otempo.com.br/economia/subscription-required-7.5927739?aId=1.2224441\textgreater. Acesso em: 15 Out. 2019. \\

%Existem diversas formas de citação observe os exemplos :

%\begin{itemize}
%    \item \cite{UML:JACOBSON} | \cite{POWELL:2006} \\ 
%        \cite{SCRUMGUIDE:2013} | \cite{urani1994} |\\
%        \cite{ETAL5} | \cite{ETAL4}
%
%    \item \citeonline{UML:JACOBSON} | \citeonline{POWELL:2006} \\
%        \citeonline{SCRUMGUIDE:2013} | \citeonline{urani1994} | \\
%        \citeonline{ETAL5} | \citeonline{ETAL4}
%
 %   \item \citeauthoronline{UML:JACOBSON}| \citeauthoronline{POWELL:2006} \\
 %       \citeauthoronline{SCRUMGUIDE:2013} | \citeauthoronline{urani1994} | \\
 %       \citeauthoronline{ETAL5} | \citeauthoronline{ETAL4}
%
 %   \item \citeauthor{UML:JACOBSON}| \citeauthor{POWELL:2006} \\
 %       \citeauthor{SCRUMGUIDE:2013}| \citeauthor{urani1994} | \\
 %       \citeauthor{ETAL5} | \citeauthor{ETAL4}
 %       
 %   \item \url{http://mirrors.ibiblio.org/CTAN/macros/latex/contrib/abntex2/doc/abntex2cite-alf.pdf}
%\end{itemize}
%
%Os dados devem ser definidos corretamente nos arquivos \textquote{.BIB} para a correta formatação no texto e na lista de referências.
%
%Palavras que devem ser apresentadas no glossário devem ser citadas especificamente no texto utilizando os comandos de glossário como : \gls{tag}
%
%
%Abreviaturas podem ser referenciadas diretamente 
%na versão reduzida \textquote{\acs{ifsp}} \space  
%ou longa \textquote{\acl{ifsp}}
%
%\begin{itemize}
%    \item Autor com diversas publicações no mesmo ano : \url{https://github.com/abntex/biblatex-abnt/issues/20}
%\end{itemize}


\subsection{Listas}

Em uma lista de itens cada item deve ser terminado por ponto e virgula, exceto o ultimo item que deve ter um ponto final.

\begin{itemize}
\item item 1;
\item item 2;
\item item ..;
\item item final.
\end{itemize}

\subsection{Elementos não textuais}

Elementos não textuais são aqueles que auxiliam o entendimento, não podem ficar "jogados" no texto, devem ser citados, cada elemento deve ser identificado por um \emph{label} único que permite a sua referencia, no texto utilizando \emph{ref} ou \emph{autoref}, esses elementos quando definidos corretamente também são inseridos nas listas presentes antes do sumário.

Lembre que o \LaTeX \  vai posicionar os elementos  da melhor maneira possível dentro do documento, sempre faça as referencias utilizando os comandos específicos, nunca utiliza "acima", "baixo", "a seguir", etc...

O posicionamento desses elementos é feito pelas rotinas do pacote float, leia a documentação em  \url{http://linorg.usp.br/CTAN/macros/latex/contrib/float/float.pdf}. A opção H deverá ser utilizada somente como ultima alternativa de posicionamento.





\input{exemplos/tabelas}
\input{exemplos/figuras}
\input{exemplos/todo}
\input{exemplos/qrcode}
\input{exemplos/exemploA3}

