
% ----------------------------------------------------------
% Introdução (exemplo de capítulo sem numeração, mas presente no Sumário)
% ----------------------------------------------------------
\chapter[Introdução]{Introdução}
Atualmente, o mercado de jogos eletrônicos movimenta acima de 130 bilhões de dolares ao redor do planeta, como pode ser visto em pesquisas publicadas pela \citeonline{artigosNewZoo}, empresa especializada em coleta e estudo de dados do mercado de jogos digitais, e nos artigos da \citeonline{artigoPUC} e \citeonline{artigoOTempo}. Dentro do mercado de jogos, existe um nicho de destaque, que consiste no mercado de Skins, que são equipamentos ou customizações que podem ser compradas com dinheiro físico e utilizadas dentro do jogo especifico que as detêm. Tal nicho vem crescendo fortemente, como diz artigo \citeonline{materiaEArena}.

Mesmo sendo um mercado forte, movimentando mais 10 bilhões de dólares por ano no jogo ‘Counter-Strike: Global Offensive’, como dito na máteria ‘Mercado de skins de CS:GO pode movimentar até 10 bilhões de dólares por ano’ do portal \citeonline{materiaTheEnemy}, as aplicações criadas para este nicho exploraram poucas vertentes de mercado até então, sendo baseadas em simples sistemas de mercados de Skins, em que um vendedor oferece sua mercadoria para que algum comprador interessado faça negócio, ou sendo baseadas em jogos de azar.

Como este mercado gera muito capital, como dito anteriormente, uma grande quantidade de aplicações, principalmente web, foram implementadas e estão no ar, o que saturou os softwares que seguem as vertentes citadas anteriormente. Com isso, como seria possível criar uma aplicação web para este mercado de skins de forma que tal sistema não caia em saturação?

Com objetivo de resolver este problema, este trabalho de pesquisa consistirá no desenvolvimento de um software web para venda de skins de jogos eletrônicos, neste caso levando em conta skins de ‘Counter-Strike: Global Offensive’, que terá como funcionamento um sistema de bolsa de valores e Day Trading, saindo do simples mercado já bem explorado.

\section{Questão de Pesquisa}
O tema deste projeto de pesquisa consiste no desenvolvimento de aplicações web para venda de skins de jogos eletrônicos.

A problemática do projeto de pesquisa reside na implementação de uma aplicação de venda de skins do jogo ‘Counter-Strike: Global Offensive’ que funcione como um sistema de bolsa de valores e Day Trading, simulando um mercado financeiro de skins, saindo da visão saturada de mercado comum, onde um vendedor simplesmente anuncia sua mercadoria, e um comprador a compra diretamente.

\section{Objetivos}
Em linhas gerais, o objetivo deste projeto é desenvolver uma solução inovadora no mercado de venda de skins de CS:GO que implementará um sistema de bolsa de valores e Day Trading dessas skins, algo não visto no mercado até então.

\subsection{Objetivos Primários}
Os objetivos principais giram em torno de elaborar e desenvolver um novo sistema de venda de skins de jogos eletrônicos, inovando o cenário atual. Para isso, será necessário conhecer as soluções existentes no mercado quando falamos em aplicações de venda de skins de jogos e entender seu funcionamento como um todo, para que seja possível identificar os pontos fortes e fracos deste mercado atualmente.

\subsection{Objetivos Secundários}
Como objetivos secundários, o projeto visa promover estudo e capacitação sobre tecnologias e meios utilizados atualmente para a implementação de aplicações web, além de promover um melhor entendimento sobre os ideais de mercado financeiro de ações e sobre o mecanismo de Day Trading.

\section{Justificativa}
As principais razões que se levam à execução deste projeto e, por consequência, ao desenvolvimento de aplicações web para venda de skins de jogos eletrônicos são: 

\begin{itemize}
	\item Inovação do mercado, pois a implementação de um mercado de ações e um Day Trade de skins de CSGO consiste em algo novo para o cenário desenvolvido até então, que permeia sites de mercado comum e aposta;
	\item Exploração de novas tecnologias que farão parte do desenvolvimento do sistema como Handlebars, por exemplo, possibilitando estudo prático para os membros do projeto;
	\item Viabilidade comercial, pois se pode lucrar muito com taxas de trocas, como pode ser visto no artigo de pesquisa da \citeonline{artigosNewZoo} sobre o lucro do mercado de jogos em 2018, que passou de 130 bilhões de dólares ao redor do planeta, sendo que o mercado de skins de CSGO movimenta por ano mais de 10 bilhões de dólares, segundo a matéria da \citeonline{materiaTheEnemy}.
\end{itemize} 

\chapter{Modelo Teórico e Pressupostos (ou Hipóteses) da Pesquisa}
No mercado em geral, principalmente no Brasil, não existem sites com a proposta da valorização de skins, muito menos aplicações inspiradas no mundo financeiro de bolsa de valores e Day Trading, então o desenvolvimento do sistema proposto neste documento preencherá esta lacuna tornando possível um usuário, além de fazer as suas trocas, ter uma experiência de bolsa de valores com seus itens do jogo. 

Tendo em vista que grande parte dos jogadores também utilizam suas skins para uso comercial e como forma de obter capital, foi levantada a hipótese de que o site de valorização tem grande possibilidade de obter popularidade e o agrado do público do jogo envolvido.

%---------------------------------------------------------------------------------------
\chapter{Metodologia do Projeto}
Buscando a implementação de aplicação web proposta, este projeto utilizará dos métodos ágeis, principalmente seguindo a figura da metodologia XP. Para auxiliar na execução destes métodos, a ferramenta Pivotal Tracker será utilizada para administrar as histórias de usuários e promover um ambiente para organizá-las e discuti-las.

Como parte dos métodos ágeis, o sistema utilizará duas ferramentas de testes, o Jest e o Cucumber, realizando testes unitários, de integração e testes End-to-End, além do fato de que o Cucumber fará os testes direto nos cenários criados pelas histórias de usuário.

Em questão técnica, a aplicação será feita em cima de um back-end em Node.Js, enquanto o front-end utilizará do HTML e do Handlebars. Para controle de versão, o Git será utilizando, portanto o projeto contém um repositório no GitHub onde cada progresso será salvo, controlando o ciclo de vida do software. Para deploy, o Heroku será utilizado. Com isso, uma aplicação foi criada no Heroku, onde o sistema será posto no ar toda vez que uma mudança for ‘commitada’ no GitHub.

Para estudo de domínio, informações com players ativos de Counter-Strike: Global Offensive serão coletadas, além de informações sobre sites de grande importância no meio de mercado de skins. Como não é uma área documentada, será necessário fazer uma abordagem mais informal e retirar informações direto de clientes e de experiencias próprias dos integrantes deste trabalho.

Pesquisas com alguns jogadores para verificar a viabilidade desta aplicação proposta no projeto serão realizadas.

