% ---
% Conclusão (outro exemplo de capítulo sem numeração e presente no sumário)
% ---
\chapter*[Conclusão]{Conclusão}
\addcontentsline{toc}{chapter}{Conclusão}
Em conclusão, a aplicação foi construída usando Node.js como back-end e HTML5 como front-end. As dificuldades encontradas nesta fase foram causadas por conta da falta de conhecimento em back-end e em Node.js, o que ocasionou a utilização do template engine Handlebars. Contudo, o aprendizado foi consideravelmente rápido, o que possibilitou a finalização das mecânicas principais prometidas ao início do projeto.

O GitHub foi utilizado como gerenciador de versão e não levantou grandes problemas ou mistérios. Em geral, a utilização foi feita de forma correta, versionando apenas os arquivos essenciais e excluindo outros, como o node\_modules. Porém, a documentação do projeto foi guardada no mesmo repositório que o código, o que não foi uma boa prática. Por mais que não estrague o repositório, o ideal seria utilizar um segundo.

Como ferramenta de deploy, o Heroku foi utilizado e nenhum problema foi encontrado em sua utilização, pois sua documentação foi de bom tamanho para um aprendizado rápido e boa utilização da ferramenta. O addon do PostgreSQL do Heroku foi utilizado e tal demandou uma pesquisa mais aprofundada, pois a documentação por si só não se identificou suficiente para sua utilização de maneira limpa. Todavia, após poucas provas de conceito e algumas pesquisas, sua utilização foi efetiva.

O desenvolvimento do projeto foi feito em sprints, o que foi totalmente novo para a equipe, que se adaptou bem à metodologia de desenvolvimento e conseguiu realizar o projeto a tempo, dividindo razoavelmente bem as tarefas. 

Quanto as tarefas, é importante ressaltar que suas escolhas, definições e mensurações foram bem discutidas, o que foi bem crucial para a implementação do projeto, pois culminaram na simplificação das User Stories, exclusão de algumas e troca de ideias em relação a que tecnologias seriam utilizadas para criação do sistema.

% ---
