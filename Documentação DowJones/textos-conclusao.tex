% ---
% Conclusão (outro exemplo de capítulo sem numeração e presente no sumário)
% ---
\chapter*[Conclusão]{Conclusão}
\addcontentsline{toc}{chapter}{Conclusão}
Em suma, o projeto construído consiste em uma aplicação web para vendas de skins do jogo 
Counter-Strike: Global Offensive. Node.js foi utilizado para o desenvolvimento back-end,
enquanto o HTML5 foi utilizado para o front-end. Como template-engine, o handlebars foi 
utilizado, tornando a aplicação server-side.
Como metodologia de engenharia de software, os métodos ágeis foram aplicados, e o 
desenvolvimento do sistema foi feito em cima de Users Stories separadas por sprints, 
similarmente a metodologia XP.  
Para a implementação do projeto, duas APIs da Steam foram utilizadas, sendo uma para 
autenticação, e outra para resgate de inventário. O GitHub foi utilizado para como 
gerenciador de versão, enquanto o Heroku foi utilizado como ferramenta de deploy.
Além disso, foi utilizado o PostgreSQL do Heroku como banco de dados da aplicação, salvando 
assim as skin, as contas dos usuários, seus inventários e seus históricos de venda.
Como testes, testes unitários e de integração foram implementados em grande quantidade, 
enquanto um único teste end-to-end automatizado foi desenvolvido. A ferramenta de análise 
estática ‘jshint’ foi utilizada para neutralização de code smells.
Ao final de todo o desenvolvimento, a aplicação construída foi um sucesso, abraçando 
mecânicas de depósito, retirada, compra, valorização e investimento de skins, além de ter um 
mecanismo de carteira. Todos esses mecanismos feitos de forma simulada, não realizando a 
real troca ou compra de skins, assim como deposito e saque de dinheiro da carteira.

% ---
