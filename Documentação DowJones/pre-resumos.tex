% ---
% RESUMOS
% ---

% resumo em português
\setlength{\absparsep}{18pt} % ajusta o espaçamento dos parágrafos do resumo
\begin{resumo}
Este trabalho de pesquisa visa a implementação de uma aplicação web que simule um
mercado financeiro para skins presentes no jogo eletrônico ``Counter-Strike: Global
Offensive'', funcionando como uma bolsa de valores e permitindo a modalidade de
negócio ``Day Trading''. Neste sistema, o usuário irá logar via Steam Account e
depositará suas skins no site, permitindo assim o investimento de suas skins, assim como
a compra de outras skins em investimento. O desenvolvimento e planejamento do
projeto beberá dos métodos ágeis, aplicando uma engenharia de software próxima ao
que reside na metodologia XP. Para isso a ferramenta Pivotal Tracker será utilizada
para administração do projeto, onde todas as Histórias de Usuário serão organizadas e
discutidas. Partindo para a parte mais técnica, o sistema será implementado utilizando
``HTML5'' para o front-end e ``Node.Js'' para o back-end. No back-end, o framework
``express'' será responsável por instanciar e administrar o servidores web. Para testes, os
frameworks ``Jest'' e ``Cucumber'' serão utilizados, enquanto para deploy e controle de
versão, o Heroku e o Github serão utilizados, respectivamente.


 \textbf{Palavras-chaves}:day trading de skins. bolsa de valores de skins. sistema de troca de
skins. aplicação web.
\end{resumo}

% resumo em inglês
\begin{resumo}[Abstract]
 \begin{otherlanguage*}{english}
%   This is the english abstract.
   \vspace{\onelineskip}

This research work aims at the implementation of a web application that simulates a
financial market for skins in the electronic game ``Counter-Strike: Global
Offensive'', functioning as a stock exchange and allowing the modality of
Day Trading business. In this system, the user will login via Steam Account and
deposit your skins on the site, allowing you to invest your skins as well as
buying other investment skins. The development and planning of the
project will drink from the applicable methods by applying software engineering
which lies in the XP methodology. For this the Pivotal Tracker tool will be used
for project administration, where all the ``User Stories'' will be organized and
discussed. Starting with the most technical part, the system will be implemented using
HTML5 for the front end and Node.Js for the back end. In the backend, the framework
Express will be responsible for instantiating and administering the web server. For testing, the
``Jest'' and ``Cucumber'' frameworks will be used, while deploying and controlling
version, Heroku and Github will be used respectively.

 
   \noindent 
   \textbf{Key-words}: skins day trading. skins stock exchange. skins exchange system. web application.
 \end{otherlanguage*}
\end{resumo}